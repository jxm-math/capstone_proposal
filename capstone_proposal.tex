%%%%%%%%%%%%%%%%%%%%%%%%%%%%
%Personal Cover Letter for Udacity - Juan Jos� Madrigal
%%%%%%%%%%%%%%%%%%%%%%%%%%%%

%%%%%% Packages

\documentclass[11pt]{article}
\usepackage[latin1]{inputenc}
\usepackage{graphicx}
\usepackage{hyperref}
\usepackage[none]{hyphenat}
\hypersetup{colorlinks,urlcolor=light}
\usepackage{multicol}
\usepackage{array}
\usepackage{cancel}
\usepackage{color}
\usepackage{amssymb}
\usepackage{amsopn}
\usepackage{amsmath}
\usepackage{enumitem}
\usepackage[default,osfigures,scale=0.95]{opensans}
\setlength{\hoffset}{-1.85cm}
\setlength{\voffset}{-2.5cm}
\setlength{\textwidth}{469pt}
\setlength{\textheight}{640pt}
\setlength{\parindent}{0cm}
\definecolor{dark}{rgb}{0.15,0.30,0.4}
\definecolor{light}{rgb}{0.15,0.3,0.6}

%%%%%% Document

\begin{document}
	
	\thispagestyle{empty} % No numbering
	
	%%% Header
	
	\begin{center}
		{{\Large\textsc{\color{dark}Machine Learning Engineer Nanodegree}}}\\
		\vspace{0.3cm}
		{{\LARGE\textsc{\color{dark}Capstone Proposal}}}
	\end{center}
	
	\vspace{-0.4cm}
	
	\textcolor{dark}{\rule{\textwidth}{3pt}}
	
	%%% Personal data
	
	\begin{minipage}[t]{7cm}
		\flushleft
		Juan Jos� Madrigal Mart�nez\\
		January 10, 2017\\
	\end{minipage}
	\hfill
	%%% Contact
	\begin{minipage}[t]{7cm}
		\flushright
		Madrid, Spain\\
		(+0034) 600 86 32 48\\
		juanjomadrigal326@gmail.com\\
		\href{https://es.linkedin.com/in/juan-jose-madrigal}{LinkedIn} \quad
		\href{https://github.com/jxm-math}{GitHub} \quad
		\href{https://www.kaggle.com/jxmmath}{Kaggle} \quad
		\href{https://discussions.udacity.com/users/juanjo_madrigal/summary}{Udacity}
	\end{minipage}\\\\
	
	\vspace{0.3cm}

%%% Domain Background

{{\Large\textsc{\color{dark}Domain Background}}}

\vspace{-0.25cm}

\textcolor{dark}{\rule{\linewidth}{2pt}} \vspace{-0.5cm}

% (approx. 1-2 paragraphs)

This project aims at building a video analysis system for surveillance purposes. It basically finds and indexes the movement-events happening in the video source (assuming a fixed camera).\\

%In this section, provide brief details on the background information of the domain from which the project is proposed. Historical information relevant to the project should be included. It should be clear how or why a problem in the domain can or should be solved. Related academic research should be appropriately cited in this section, including why that research is relevant. Additionally, a discussion of your personal motivation for investigating a particular problem in the domain is encouraged but not required.\\

%%% Problem Statement

{{\Large\textsc{\color{dark}Problem Statement}}}

\vspace{-0.25cm}

\textcolor{dark}{\rule{\linewidth}{2pt}} \vspace{-0.5cm}

% (approx. 1 paragraph)

In video surveillance context, a common problem is having to explore large amounts of videos looking for a particular event, e.g., someone getting in or doing something in particular. Users have to deal with a sequential (and manual) search through the video-source (which can be many hours of stored video) and this is rather inefficient, boring and error prone.\\

%In this section, clearly describe the problem that is to be solved. The problem described should be well defined and should have at least one relevant potential solution. Additionally, describe the problem thoroughly such that it is clear that the problem is quantifiable (the problem can be expressed in mathematical or logical terms) , measurable (the problem can be measured by some metric and clearly observed), and replicable (the problem can be reproduced and occurs more than once).\\

%%% Datasets and Inputs

{{\Large\textsc{\color{dark}Datasets and Inputs}}}

\vspace{-0.25cm}

\textcolor{dark}{\rule{\linewidth}{2pt}} \vspace{-0.5cm}

% (approx. 2-3 paragraphs)

In this section, the dataset(s) and/or input(s) being considered for the project should be thoroughly described, such as how they relate to the problem and why they should be used. Information such as how the dataset or input is (was) obtained, and the characteristics of the dataset or input, should be included with relevant references and citations as necessary It should be clear how the dataset(s) or input(s) will be used in the project and whether their use is appropriate given the context of the problem.\\

%%% Solution Statement

{{\Large\textsc{\color{dark}Solution Statement}}}

\vspace{-0.25cm}

\textcolor{dark}{\rule{\linewidth}{2pt}} \vspace{-0.5cm}

% (approx. 1 paragraph)

We propose the use of unsupervised clustering techniques to find the movement-events in the movement-mask space (binary representation: 1 for pixel-moved, 0 for pixel-no-moved). The latter is computed with a foreground subtraction step resulting in a $\texttt{M}\times \texttt{N}\times \texttt{T}$ binary array, for a $\texttt{M}\times \texttt{N}$ video source with $\texttt{T}$ frames. Every movement event is represented in this 3D space as a 'volume' or 'body' (contiguous '1' cells of movements), and the unsupervised clustering algorithm is supposed to successfully segment most of them. A web application is envisaged to be built, receiving video files and delivering the list of detected events linked to the corresponding point in the original video source.\\

%In this section, clearly describe a solution to the problem. The solution should be applicable to the project domain and appropriate for the dataset(s) or input(s) given. Additionally, describe the solution thoroughly such that it is clear that the solution is quantifiable (the solution can be expressed in mathematical or logical terms) , measurable (the solution can be measured by some metric and clearly observed), and replicable (the solution can be reproduced and occurs more than once).\\


%%% Benchmark Model

{{\Large\textsc{\color{dark}Benchmark Model}}}

\vspace{-0.25cm}

\textcolor{dark}{\rule{\linewidth}{2pt}} \vspace{-0.5cm}

% (approximately 1-2 paragraphs)

In this section, provide the details for a benchmark model or result that relates to the domain, problem statement, and intended solution. Ideally, the benchmark model or result contextualizes existing methods or known information in the domain and problem given, which could then be objectively compared to the solution. Describe how the benchmark model or result is measurable (can be measured by some metric and clearly observed) with thorough detail.\\

%%% Evaluation Metrics

{{\Large\textsc{\color{dark}Evaluation Metrics}}}

\vspace{-0.25cm}

\textcolor{dark}{\rule{\linewidth}{2pt}} \vspace{-0.5cm}

% (approx. 1-2 paragraphs)

In this section, propose at least one evaluation metric that can be used to quantify the performance of both the benchmark model and the solution model. The evaluation metric(s) you propose should be appropriate given the context of the data, the problem statement, and the intended solution. Describe how the evaluation metric(s) are derived and provide an example of their mathematical representations (if applicable). Complex evaluation metrics should be clearly defined and quantifiable (can be expressed in mathematical or logical terms).\\

%%% Project Design

{{\Large\textsc{\color{dark}Project Design}}}

\vspace{-0.25cm}

\textcolor{dark}{\rule{\linewidth}{2pt}} \vspace{-0.5cm}

% (approx. 1 page)

In this final section, summarize a theoretical workflow for approaching a solution given the problem. Provide thorough discussion for what strategies you may consider employing, what analysis of the data might be required before being used, or which algorithms will be considered for your implementation. The workflow and discussion that you provide should align with the qualities of the previous sections. Additionally, you are encouraged to include small visualizations, pseudocode, or diagrams to aid in describing the project design, but it is not required. The discussion should clearly outline your intended workflow of the capstone project.\\



%Before submitting your proposal, ask yourself. . .

%Does the proposal you have written follow a well-organized structure similar to that of the project template?

%Is each section (particularly Solution Statement and Project Design) written in a clear, concise and specific fashion? Are there any ambiguous terms or phrases that need clarification?

%Would the intended audience of your project be able to understand your proposal?

%Have you properly proofread your proposal to assure there are minimal grammatical and spelling mistakes?

%Are all the resources used for this project correctly cited and referenced?
	
\end{document}
